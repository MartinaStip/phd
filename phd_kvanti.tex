% Options for packages loaded elsewhere
% Options for packages loaded elsewhere
\PassOptionsToPackage{unicode}{hyperref}
\PassOptionsToPackage{hyphens}{url}
\PassOptionsToPackage{dvipsnames,svgnames,x11names}{xcolor}
%
\documentclass[
  czech,
  14pt,
  a4paper,
  DIV=11,
  numbers=noendperiod]{scrreprt}
\usepackage{xcolor}
\usepackage[top=2.5cm, bottom=2.5cm, left=2.5cm, right=2.5cm,
footskip=1.5cm]{geometry}
\usepackage{amsmath,amssymb}
\setcounter{secnumdepth}{5}
\usepackage{iftex}
\ifPDFTeX
  \usepackage[T1]{fontenc}
  \usepackage[utf8]{inputenc}
  \usepackage{textcomp} % provide euro and other symbols
\else % if luatex or xetex
  \usepackage{unicode-math} % this also loads fontspec
  \defaultfontfeatures{Scale=MatchLowercase}
  \defaultfontfeatures[\rmfamily]{Ligatures=TeX,Scale=1}
\fi
\usepackage{lmodern}
\ifPDFTeX\else
  % xetex/luatex font selection
  \setmainfont[]{Franklin Gothic Book}
  \setsansfont[]{Franklin Gothic Book}
  \setmonofont[]{Franklin Gothic Book}
\fi
% Use upquote if available, for straight quotes in verbatim environments
\IfFileExists{upquote.sty}{\usepackage{upquote}}{}
\IfFileExists{microtype.sty}{% use microtype if available
  \usepackage[]{microtype}
  \UseMicrotypeSet[protrusion]{basicmath} % disable protrusion for tt fonts
}{}
\makeatletter
\@ifundefined{KOMAClassName}{% if non-KOMA class
  \IfFileExists{parskip.sty}{%
    \usepackage{parskip}
  }{% else
    \setlength{\parindent}{0pt}
    \setlength{\parskip}{6pt plus 2pt minus 1pt}}
}{% if KOMA class
  \KOMAoptions{parskip=half}}
\makeatother
% Make \paragraph and \subparagraph free-standing
\makeatletter
\ifx\paragraph\undefined\else
  \let\oldparagraph\paragraph
  \renewcommand{\paragraph}{
    \@ifstar
      \xxxParagraphStar
      \xxxParagraphNoStar
  }
  \newcommand{\xxxParagraphStar}[1]{\oldparagraph*{#1}\mbox{}}
  \newcommand{\xxxParagraphNoStar}[1]{\oldparagraph{#1}\mbox{}}
\fi
\ifx\subparagraph\undefined\else
  \let\oldsubparagraph\subparagraph
  \renewcommand{\subparagraph}{
    \@ifstar
      \xxxSubParagraphStar
      \xxxSubParagraphNoStar
  }
  \newcommand{\xxxSubParagraphStar}[1]{\oldsubparagraph*{#1}\mbox{}}
  \newcommand{\xxxSubParagraphNoStar}[1]{\oldsubparagraph{#1}\mbox{}}
\fi
\makeatother


\usepackage{longtable,booktabs,array}
\usepackage{calc} % for calculating minipage widths
% Correct order of tables after \paragraph or \subparagraph
\usepackage{etoolbox}
\makeatletter
\patchcmd\longtable{\par}{\if@noskipsec\mbox{}\fi\par}{}{}
\makeatother
% Allow footnotes in longtable head/foot
\IfFileExists{footnotehyper.sty}{\usepackage{footnotehyper}}{\usepackage{footnote}}
\makesavenoteenv{longtable}
\usepackage{graphicx}
\makeatletter
\newsavebox\pandoc@box
\newcommand*\pandocbounded[1]{% scales image to fit in text height/width
  \sbox\pandoc@box{#1}%
  \Gscale@div\@tempa{\textheight}{\dimexpr\ht\pandoc@box+\dp\pandoc@box\relax}%
  \Gscale@div\@tempb{\linewidth}{\wd\pandoc@box}%
  \ifdim\@tempb\p@<\@tempa\p@\let\@tempa\@tempb\fi% select the smaller of both
  \ifdim\@tempa\p@<\p@\scalebox{\@tempa}{\usebox\pandoc@box}%
  \else\usebox{\pandoc@box}%
  \fi%
}
% Set default figure placement to htbp
\def\fps@figure{htbp}
\makeatother



\ifLuaTeX
\usepackage[bidi=basic]{babel}
\else
\usepackage[bidi=default]{babel}
\fi
\ifPDFTeX
\else
\babelfont{rm}[]{Franklin Gothic Book}
\fi
% get rid of language-specific shorthands (see #6817):
\let\LanguageShortHands\languageshorthands
\def\languageshorthands#1{}


\setlength{\emergencystretch}{3em} % prevent overfull lines

\providecommand{\tightlist}{%
  \setlength{\itemsep}{0pt}\setlength{\parskip}{0pt}}



 


% Set main sans-serif font with italic support
\setsansfont{Franklin Gothic Book}[
  ItalicFont = Franklin Gothic Book Italic,
  BoldFont = Franklin Gothic Medium,
  BoldItalicFont = Franklin Gothic Medium Italic
]

% Make Franklin Gothic the default family everywhere
\renewcommand{\familydefault}{\sfdefault}

% Headings
\setkomafont{chapter}{\bfseries\fontspec{Franklin Gothic Medium}}
\setkomafont{section}{\bfseries\fontspec{Franklin Gothic Medium}}
\setkomafont{subsection}{\bfseries\fontspec{Franklin Gothic Medium}}
\setkomafont{subsubsection}{\bfseries\fontspec{Franklin Gothic Medium}}

% Title block
\setkomafont{title}{\bfseries\fontspec{Franklin Gothic Medium}}
\setkomafont{subtitle}{\bfseries\fontspec{Franklin Gothic Medium}}

% Page headers and footers
\setkomafont{pagehead}{\bfseries\fontspec{Franklin Gothic Medium}}
\setkomafont{pagenumber}{\fontspec{Franklin Gothic Book}}

% Captions
\setkomafont{caption}{\bfseries\fontspec{Franklin Gothic Medium}}
\setkomafont{captionlabel}{\bfseries\fontspec{Franklin Gothic Book}}
\usepackage{caption}
\captionsetup[figure]{skip=3pt}

% TOC (use tocloft instead of KOMA fonts)
\usepackage{tocloft}
\renewcommand{\cftchapfont}{\bfseries\fontspec{Franklin Gothic Medium}}
\renewcommand{\cftsecfont}{\fontspec{Franklin Gothic Book}}
\renewcommand{\cftsubsecfont}{\fontspec{Franklin Gothic Book}}

% space between top of page and heading
\renewcommand*{\chapterheadstartvskip}{\vspace*{0pt}}
\KOMAoption{captions}{tableheading,figureheading}
\makeatletter
\@ifpackageloaded{caption}{}{\usepackage{caption}}
\AtBeginDocument{%
\ifdefined\contentsname
  \renewcommand*\contentsname{Obsah}
\else
  \newcommand\contentsname{Obsah}
\fi
\ifdefined\listfigurename
  \renewcommand*\listfigurename{Seznam obrázků}
\else
  \newcommand\listfigurename{Seznam obrázků}
\fi
\ifdefined\listtablename
  \renewcommand*\listtablename{Seznam tabulek}
\else
  \newcommand\listtablename{Seznam tabulek}
\fi
\ifdefined\figurename
  \renewcommand*\figurename{Graf}
\else
  \newcommand\figurename{Graf}
\fi
\ifdefined\tablename
  \renewcommand*\tablename{Tabulka}
\else
  \newcommand\tablename{Tabulka}
\fi
}
\@ifpackageloaded{float}{}{\usepackage{float}}
\floatstyle{ruled}
\@ifundefined{c@chapter}{\newfloat{codelisting}{h}{lop}}{\newfloat{codelisting}{h}{lop}[chapter]}
\floatname{codelisting}{Výpis}
\newcommand*\listoflistings{\listof{codelisting}{Seznam výpisů}}
\makeatother
\makeatletter
\makeatother
\makeatletter
\@ifpackageloaded{caption}{}{\usepackage{caption}}
\@ifpackageloaded{subcaption}{}{\usepackage{subcaption}}
\makeatother
\usepackage{bookmark}
\IfFileExists{xurl.sty}{\usepackage{xurl}}{} % add URL line breaks if available
\urlstyle{same}
\hypersetup{
  pdfauthor={Martina Štípková; Gabriela Fatková; František Kalvas},
  pdflang={cs},
  colorlinks=true,
  linkcolor={blue},
  filecolor={Maroon},
  citecolor={Blue},
  urlcolor={Blue},
  pdfcreator={LaTeX via pandoc}}


\title{\includegraphics[width=7cm,height=\textheight,keepaspectratio]{odbor_kvality.png}\\
\strut \\
\strut \\
\strut \\
Šetření studujících v doktorských programech 2024/25}
\author{Martina Štípková \and Gabriela Fatková \and František Kalvas}
\date{}
\begin{document}
\maketitle

\renewcommand*\contentsname{Obsah}
{
\hypersetup{linkcolor=}
\setcounter{tocdepth}{2}
\tableofcontents
}

\chapter{Úvod}\label{uxfavod}

Toto je zpráva z pilotního šetření studujících v doktorských studijních
programech, které proběhlo od května do srpna 2025. Dotázáni byli
studující ve všech ročnících. Témata zahrnovala hodnocení podmínek
studia, množství povinností, rozvoje svých kompetencí, podmínek pro
tvůrčí činnost a další.

Jedná se o první šetření svého druhu, proto jsme do dotazníku zařadili
prostor pro otevřené odpovědi k jednotlivým tématickým blokům. Díky tomu
se studující mohli vyjádřit vlastními slovy v případě, že by nabízené
možnosti nepokrývaly všechny aspekty, které nám chtějí sdělit. Na konci
dotazníku odpovídající navíc dostali možnost shrnout, co na svém studiu
nejvíce oceňují, a co by naopak změnili. Kvalitativní analýza všech
otevřených odpovědí je klíčovou součástí výsledků. Při jejich
interpretaci je ale nutné mít na paměti, že vypisovat otevřený komentář
vyžaduje větší úsilí, než vybrat z nabízených možností. Toto úsilí
vyvinou zejména studující, kteří jsou s některým aspektem svého studia
velmi nespokojeni, nebo mají naopak velkou motivaci něco pochválit.

\chapter{Hlavní zjištění}\label{hlavnuxed-zjiux161tux11bnuxed}

\begin{itemize}
\item
  Klíčovou roli v doktorském studiu hrají školitelé a školitelky.
  Studující si velmi cení, když mají jejich dostatečnou podporu. Pokud
  ale školící ososby ve své roli selhávají, je to pro studující velmi
  demotivující. Pozitivní je, že studující, kteří jsou s prací svých
  školících osob výrazně, jasně převažují.
\item
  Doktorské studium je oceňováno zejména pro svobodu bádání a možnost
  rozvíjet vlastní výzkum. Studující také oceňují flexibilitu studijního
  plánu, otevřenost k jejich individuálním potřebám a dobré zázemí a
  vztahy na svém pracovišti.
\item
  Hůře hodnocené je množství povinností, slabá podpora pro osvojení
  akademických kompetencí, nejasné či proměnlivé požadavky,
  nespravedlivé rozdělení pracovní zátěžě a nedostatečné finanční
  ohodnocení. Tyto aspekty vedou k nejistotě a snížené motivaci k
  dokončení studia.
\item
  Problematické jevy, které ohrožují sociální bezpečí, se dějí velmi
  zřídka.
\end{itemize}

\chapter{Návranost šetření a charakteristika
vzorku}\label{nuxe1vranost-ux161etux159enuxed-a-charakteristika-vzorku}

TĚCH 551 ADRES V EMAILU S DOTAZNÍKEM, KTERÉ ZJISTIL FRANTIŠEK OD PANA
WIMMERA JE ASI DOST NADHODNOCENÉ ČÍSLO, BUDOU TAM I ADRESY LIDÍ, CO UŽ
NESTUDUJÍ. NA STATISTIKÁCH MAJÍ LENKY 448 STUDUJÍCÍCH PHD KE KONCI
PROSINCE 2025. VYTÁHLA JSEM Z KOSTKY STAV K POSLEDNÍMU KVĚTNU, COŽ CCA
ODPOVÍDÁ ZADÁNÍ NAŠEHO VÝZKUMU A VYŠLO MI 411. PRACUJU S TÍM, ALE JEŠTĚ
JE TŘEBA SI TO NECHAT ZKONTROLOVAT.

Šetření mělo bohužel velmi malou návratnost. Z 411 studujících, kteří
byli na konci května 2025 zapsáni v doktorských programech, dotazník
alespoň částečně vyplnilo pouze \textbf{75 osob}, tj. \textbf{18 \%}. Je
možné že za nízkou návraností stojí načasování šetření na konci
akademického roku (květen -- srpen 2025). Dalším možným vysvětlením
mohou být obavy dotazovaných z poskytování zpětné vazby. Všechny otázky
byly definovány jako dobrovolné (odpovídající je mohli přeskočit).
Přesto se ale mohli obávat, že by si vedení jejich pracoviště dokázalo
spojit odpovědi s jejich osobou, už jen kvůli celkově malému počtu
doktorských studujících na fakultách. Tomu odpovídá skutečnost, že
naprostá většina studujících neuvedla svůj studijní program. Další
klasifikační údaje (fakulta, gender) neuvedlo kolem 20 \%
odpovídajících. Na věcné otázky velká většina odpověděla.

Otevřené komentáře k průběhu svého doktorského studia máme ze všech
fakult kromě FPR. Nejsilnější nestrukturovaná zpětná vazba je z FAV, což
odpovídá tomu, že na této fakultě je nejvíce doktorských studujících.

\begin{figure}[H]

\caption{\label{fig-response}}

\centering{

\pandocbounded{\includegraphics[keepaspectratio]{figs/response.png}}

}

\end{figure}%

\begin{figure}[H]

\caption{\label{fig-response_fak}}

\centering{

\pandocbounded{\includegraphics[keepaspectratio]{figs/response_fak.png}}

}

\end{figure}%

\begin{figure}[H]

\caption{\label{fig-response_gender}}

\centering{

\pandocbounded{\includegraphics[keepaspectratio]{figs/response_gender.png}}

}

\end{figure}%

\chapter{Charakteristika a podmínky studia - kvantitativní
výsledky}\label{charakteristika-a-podmuxednky-studia---kvantitativnuxed-vuxfdsledky}

\begin{figure}[H]

\caption{\label{fig-study}}

\centering{

\pandocbounded{\includegraphics[keepaspectratio]{figs/study.png}}

}

\end{figure}%

\begin{figure}[H]

\caption{\label{fig-condition}}

\centering{

\pandocbounded{\includegraphics[keepaspectratio]{figs/condition.png}}

}

\end{figure}%

\begin{figure}[H]

\caption{\label{fig-competences}}

\centering{

\pandocbounded{\includegraphics[keepaspectratio]{figs/competence.png}}

}

\end{figure}%

\begin{figure}[H]

\caption{\label{fig-research}}

\centering{

\pandocbounded{\includegraphics[keepaspectratio]{figs/research.png}}

}

\end{figure}%

\begin{figure}[H]

\caption{\label{fig-finance}}

\centering{

\pandocbounded{\includegraphics[keepaspectratio]{figs/finance.png}}

}

\end{figure}%

SEM BUĎ KOMPLET KVALI ANALÝZU, NEBO JI NAPORCOVAT A PŘIDAT KE KONKRÉTNÍM
GRAFŮM. ZVÁŽILA BYCH OMEZENÍ POČTU CITACÍ, KDYŽ SE OBSAH PŘEKRÝVÁ. ZATÍM
DÁVÁM TAK, JAK JE, ABYSME VIDĚLI, JAK DLOUHÉ TO VYJDE.

\chapter{Kvalitativní výsledky}\label{kvalitativnuxed-vuxfdsledky}

\section{Didaktické nedostatky}\label{didaktickuxe9-nedostatky}

Studující upozorňují na nedostatečné vedení a nezájem některých
školitelů, což vede k pocitu osamocenosti a demotivace. Častým problémem
je nejasnost v požadavcích, měnící se podmínky zkoušek a dlouhé prodlevy
ve zpětné vazbě. V některých případech je komunikace se školitelem
dokonce vnímána jako hrubá či nedůstojná, což negativně ovlivňuje celou
atmosféru pracoviště.

\emph{Můj školitel na se mi nevěnuje, neustále mě přehlíží a cokoliv je
důležitější než moje studium. Neustále mi mění požadavky zkoušek či
nároky na publikace. Mnou poskytnutá práce na review je nechána bez
povšimnutí a od školitele nemám tak téměř žádnou zpětnou vazbu. Na
předem domluvené schůzky se kolikrát ani nedostavil bez jakékoliv
omluvy. V podstatě studuji doktorské studium zcela sám\ldots. Je to
vyčerpávájící a demotivující.}

\emph{Dále by bylo určitě vhodné stanovit přesně kolik zkoušek a jaké
publikační výsledky má student mít, protože pan školitel to určitě zcela
náhodně a může zde tak docházet k diskriminaci. Někdo dostane zkoušky 3,
jiný 5 a podobně. Pan školitel rovněž mění zadání zkoušky z jeho
předmětů, takže student dělá 1 zkoušku třeba na 3-4x, což může být pro
spoustu lidi velmi demotivující.}

\emph{Systém kontroly a postupu tvorby disertační práce je zcela
nepřiměřený. Je to myšleno tak, že moji práci školitel v podstatě
ignoruje i když mu všechny podklady poskytnu včas i formou emailu s
přílohou\ldots{} }

\emph{Školitel se chová často velmi hrubě, kdy on je přeci ten
nejchytřejší a všichni ostatní jsou hloupí. Přijde mi, že pod ním celá
sekce a lidé v ní značně trpí, protože se s ním bojí jít do konfliktu
nebo na tyto věci upozornit.}

\emph{Uvítal bych nastavení konkrétních termínů pro poskytování zpětné
vazby k seminárním pracím; to v některých případech přesahuje měsíc a
často až po opakovaném připomínání, což může působit demotivačně.}

\emph{Dále by mohl být lepší přístup školitele/školitelů, v mém případě
si myslím, že téměř chybělo vedení, které bych od školitele očekával.}

\emph{Pedagogika na škole? těžko\ldots{} motivace na phd? Žádná. Na této
škole není motivace k výuce, sorry.}

\section{Adaptace}\label{adaptace}

Začátky studia jsou často spojené s nedostatkem informací a orientace v
akademickém prostředí. Studenti se musí spoléhat hlavně sami na sebe,
nepociťují dost podpory, ani jasné zadání. Přetížení fakultními
povinnostmi ubírá čas na vlastní výzkum a disertační práci. Chybí také
dlouhodobá koncepce doktorského vzdělávání.

\emph{Na začátku studia absentovaly informace, co to je OBD, RIV, apod.
Člověk se postupně musel naučit plavat. Postupně je člověk zapojován do
více a více fakultních aktivit, výuky apod., tak na samotnou disertační
práci zbývá času velmi málo (zejména při dalších mimouniverzitních
aktivitách).}

\emph{Bohužel to u nás to funguje tak, že student je nechán téměř zcela
napospas a co si sám nezařídí, nevykomunikuje s ostatními pracovníky,
tak to nemá. Školitel nás nechává zcela bez podpory. Jediné, kdy se nám
věnuje je moment, když děláme nějakou část, která může být užitečná pro
jeho firmu\ldots{}}

\section{Byrokracie a formalistní přístup (absence
smysluplnosti)}\label{byrokracie-a-formalistnuxed-pux159uxedstup-absence-smysluplnosti}

Studující kritizují nadměrnou byrokracii a důraz na formální požadavky
na úkor smysluplného výzkumu. Povinné normy (rozsah textu, citace, dílčí
zkoušky) vnímají spíše jako překážka než přínos. Akademická kariéra je
pro ně abstraktní, preferují komerčně využitelné kompetence.

\emph{Naopak mě trápí byrokratický a akademický proces, který vyžaduje
splňovat určité normy na úkor výsledku práce (nucený počet stran práce,
hodiny konzultací, nemožnost odchýlit se od tématu, fixace na správně
udělané citace místo důrazu na praktickou aplikaci projektu). Ocenil
bych více možností setkání se spolužáky, jednodušší zapojení se do
grantů, abych nemusel tolik řešit zaměstnání.}

\emph{Nová koncepce dílčích doktorských zkoušek rovněž nevyhovuje, neboť
vytváří dojem, že se klade větší důraz na vyhovění požadavkům
zkoušejících, než na podporu vlastního výzkumného postupu}

\emph{Mám nulovou motivaci ve studiu pokračovat, protože u firem je
tento titul spíše nevýhoda než výhoda a v akademické kariéře pokračovat
nechci, protože ta práce podle mě nemá žádný smysl (např. psaní hromady
publikací, které nikdo nečte).}

\emph{Raději bych rozvíjel praktické komerční kompetence.}

\section{Tvůrčí činnost}\label{tvux16frux10duxed-ux10dinnost}

Prostor pro badatelskou činnost oceňují, slabinou však zůstává
nejednotnost v požadavcích na výstupy a někdy obtížně splnitelné nároky.

\emph{Oceňuji možnost získat nové schopnosti, praxi v publikování. Lepší
by mohla být komunikace, co přesně je po studentech požadováno k
dokončení studia (různí lidé vám řeknou různé věci např. co se týče
zahraniční stáže či požadované kvalitě konferencí/časopisů).}

\emph{S doktorským studiem jsem velmi spokojen. Jediné co se mi nelíbí
je požadavek na splnění článků Q1, Q2. Pro některé doktorandy je velmi
obtížné v dnešní době tento požadavek splnit.}

\section{Mimodidaktické vztahy}\label{mimodidaktickuxe9-vztahy}

Kvalita vztahů na pracovišti (spolupráce s kolegy, přístup školitele k
úspěchům doktorandů, vztah ke studujícím) hrají významnou roli.

\emph{Nejvíce jsem se toho naučil od ostatních pracovníku na katedře se
kterýma byla skvělá spolupráce. Ale pan školitel každý můj dílčí úspěch
nebo publikaci mimo téma zlehčoval. Stejně tak zapojení v evropských
projektech, které podle něj nemají smysl. Učení a vedení prací studentů
mě baví, ale rozhodně po dokončení studia neplánuji zůstat na
pracovišti, kde je můj školitel.}

\emph{Oceňuji mezioborové předměty a spolupráci.}

\emph{Oceňuji: různorodost práce, přátelské prostředí, ochotní kolegové}

\emph{Oceňuji, že mě výzkumní pracovníci přijali jako člena týmu.}

\emph{Oceňuji přátelskou atmosféru a své kolegy na pracovišti.}

\emph{Oceňuji začlenění na katedře, pracovní podmínky (kancelář),
podpora kolegů.}

\emph{Nulová pomoc od zkušenějších kolegů, absolutně netýmové a tím
pádem i velmi neefektivní jednání. Takové jednání škodí jak příjemné
pracovní atmosféře, tak celé fakultě. Mnoho procesů by bylo možné
optimalizovat, pokud by se více spolupracovalo.}

\section{Silné stránky
Ph.D.~studií}\label{silnuxe9-struxe1nky-ph.d.-studiuxed}

Doktorské studium je oceňováno zejména pro svobodu bádání, možnost
rozvíjet vlastní výzkum. Studenti si cení flexibility studijního plánu a
otevřenosti k individuálním potřebám.

\subsection{Svoboda bádání}\label{svoboda-buxe1duxe1nuxed}

Na doktorském studiu nejvíce oceňuji možnost věnovat se hluboce jednomu
odbornému tématu, samostatně bádat a zároveň se rozvíjet v analytickém i
kritickém myšlení. Velmi přínosná je i spolupráce s vedoucím práce a
kolegy z oboru, kteří poskytují nejen odbornou zpětnou vazbu, ale i
motivaci k dalšímu růstu. Možnost účastnit se konferencí, prezentovat
výsledky, publikovat a být součástí širší vědecké komunity považuji za
zásadní součást doktorského vzdělávání.

\emph{Oceňuji prostor pro realizaci výzkumu v mé vlastní oblasti zájmu.
Časové nároky na projekty, výuku, akce jako DoD atp. tento prostor ale
zmenšují.}

\emph{Možnost realizovat vlastní nápady a výzkum je taky velmi
podporována na naší katedře.}

\emph{Možnost a podpora rozvíjet se v tom, co mě zajímá.}

\emph{Nejvíce oceňuji přístup mého vedoucího, které po mě chce výstupy,
ale po jejich splnění mám volnou ruku pro své doktorské studium, což ne
u všech školitelů je zaručené.}

\emph{Oceňuji volnost. Práce by měla být více provázána s tématem studia
a měl by být větší tlak na termíny.}

\subsection{Individuální přístup a
flexibilita}\label{individuuxe1lnuxed-pux159uxedstup-a-flexibilita}

Gen Z -- řeší slaďování a work-life balance v\,mnohem nižším věku než
generace předešlé. Oceňuji flexibilitu studijního plánu a doby strávené
na univerzitě.

\emph{Oceňuji: přístup školitele a kolegů. Samozřejmě oceňuji
flexibilitu (možnost práce z domova, snadná domluva s vedoucím), možnost
vyjíždět na konference, do různých pracovišť a navazovat kontakty a
sdílet výzkum s vědci (ale i s lidmi z praxe) s podobným zájmem.}

\emph{Oceňuji perfektní práci studijního oddělení a že mi v průběhu roku
nikdo nehází klacky pod nohy.}

\emph{Celkově mi doktorské studium vyhovuje, ale nepovažuji za reálné
jej stihnout v základní době studia - při plnění všech dalších
povinností i snaze o rozumné živobytí. Kvalita doktorského studia se
zásadně odvíjí od role školitele, jeho erudice i tématu a zejména jeho
časových možností. Největším problémem pro mne bylo najít disertabilní
téma, a dále najít dostatek času a motivace pro dokončení studia.}

\emph{Nejvíce oceňuji čas, který mi věnuje moje vedoucí práce
(školitelka)}

\emph{Svoboda v časové organizaci. Flexibilitu oceňuji. Zlepšit by se
mohlo týmové jednání.}

\emph{Nelíbí se mi, že bych v rámci studia měl odjet na nějaký
zahraniční pobyt.}

\emph{Vstřícný přístup pracoviště, školitele a studijního oddělení}

\emph{Oceňuji přístup katedry, školitele a laboratoře při praktickém
výzkumu v rámci studia.}

\emph{Oceňuji širokou škálu možností (výuka, podílení se na přípravě
výukových materiálů, účast na propagačních akcích, atp., možnosti
výjezdu do zahraničí, na stáže, konference, apod.).}

\section{Slabé stránky
Ph.D.~studií}\label{slabuxe9-struxe1nky-ph.d.-studiuxed}

Nedostatečné finanční ohodnocení nutí studenty hledat další zaměstnání a
omezuje prostor pro výzkum. Slabá systémová podpora, roztříštěná
organizace a nejasné či proměnlivé požadavky vedou k nerovnoměrné zatěži
doktorandů, nejistotě a snížené motivaci k dokončení studia.

\subsection{Finance}\label{finance}

\emph{Co by se dalo zlepšit: finance}

\emph{Během doktorského studia je nutnost mít i zaměstnání, které je
bohužel priorita z hlediska financí.}

\emph{Doufám, že se zlepší finanční situace v oblasti doktorandských
stipendií. Přestože aktuálně nehrozí existenční riziko, bylo by žádoucí,
aby se podmínky zlepšily. V opačném případě se stávají nutnými úvahy o
práci na poloviční úvazek, což by mohlo vést k nedostatečnému času
vyhrazenému pro výzkum.}

\emph{Lepší by bylo: vyšší finanční ohodnocení v rámci měsíčního
pravidelného stipendia}

\emph{zlepšit by se mohlo financování (i retrospektivně u již
studujících)}

\subsection{Podpora}\label{podpora}

Uvítal bych větší systémovou podporu pro doktorandy -- ať už jde o
financování a jasnější kariérní perspektivy. V některých případech také
chybí koordinovanější školení v metodologii, pedagogice nebo v tzv.
``\,``soft skills''\,``, které jsou dnes pro akademické i mimoakademické
prostředí velmi důležité.

\emph{Na druhou stranu se mi nelíbí obecná organizace, tzn. absence
zaškolení (tisk, cesťáky, publikování, přístupy \ldots), absence zpětné
vazby na kvalitu mé výuky. Dále jsem si musel sám vymyslet zadání DisP.
Neexistuje žádný dlouhodobější směr výzkumu, proto na mojí DisP nikdo
nenaváže a další doktorandi budou také zanecháni svému osudu (bez
podpory a dokonce i bez zadání).}

\emph{Představoval bych si větší strukturovanost doktorských předmětů,
semináře týkající se publikační činnosti (jak fungují citace, časopisy,
akademické prostředí, financování apod.).}

\emph{Kompetence jsou rozvíjeny spíše mimochodem, než nějak
systematicky. Do výuky byl člověk hozen stylem nauč se plavat. Kurzy pro
pedagogické pracovníky ho minuly, a v nabídce se objevily až o rok
později.}

\subsection{Srozumitelnost}\label{srozumitelnost}

Pociťuji silně nerovnoměrné zatěžování doktorandů (někteří jsou
přepracovaní, jiní naopak neochotní). Je to spíše způsobené samozřejmě
přístupem. Pracovitost není oceňována ale spíše využívána.

\emph{Lepší koordinace studijního oddělení a katedry: každá strana má
jiné požadavky, občas pro studenty matoucí. Jinak jsem se studiem
spokojená, všichni mi vždy vyšli vstříc a podporovali mě ve všech
nápadech, se kterými jsem přišla.}

\emph{Bylo by potřeba zlepšit komunikaci s administrativou a zjednodušit
formuláře - online aplikace.}

\emph{Lepší by bylo neměnit podmínky k dizertaci během studia.}

\emph{Daleko lepší by měla být komunikace ohledně požadavků na studium -
jaké publikace jakého charakteru jsou vyžadovány, a to zejména u
časopiseckých publikací. V požadavcích (ze strany fakulty) není
zakotveno nic o časopiseckých publikacích, ovšem ``nepsaným pravidlem''
je vyžadovat publikaci v impaktovaném časopise indexovaném ve
WoS/Scopus. Nic proti tomuto požadavku nemám, neměl by být však pouze
nepsaným pravidlem.}

\emph{5kr. za pedagogickou činnost je každému doktorandovi udělováno za
zcela odlišnou angažovanost. Jednomu stačí připravit jednu přednášku,
ale druhý musí připravit cvičení, přednášky, zkoušky lak odpřednášet a
být u všech testů atd. Významný nepoměr.}

\section{Programy realizované
v\,angličtině}\label{programy-realizovanuxe9-v-angliux10dtinux11b}

\subsection{Podpora}\label{podpora-1}

Největším kladem studia jsou vztahy se školiteli, vyzdvihovány jsou na
prvním místě profesionální vlastnosti jako odbornost a jasné vedení,
dále pak osobní kvality jako ochota, přátelskost a spolehlivost. Stejnou
důležitost přisuzují studující i administrativnímu personálu svých
pracovišť, kteří mohou kompenzovat nedostatky ve vztahu se školitelem.
Studenti oceňují i finanční podporu ve formě stipendia.

\emph{First and foremost, I really appreciate the collaboration with my
supervisor. There were also a few other faculty members (secretary,
former Ph.D.~students, fellow Ph.D.~students) who are very friendly,
helpful, and reliable.}

\emph{I deeply appreciate the support and guidance from my supervisor,
whose expertise has been invaluable in my academic progress. I am also
grateful to my instructors and the university staff for their assistance
and collaboration. Additionally, I am sincerely thankful to the Ministry
of Education, Youth and Sport for providing the government scholarship
that makes my doctoral studies possible.}

\emph{I really appreciated the help provided by the department staff,
who did everything they could to compensate for the lack of
communication with professors.}

\emph{I love the help render to me by my supervisor and the entire staff
of the fakulty.}

\emph{Grateful for the state-of-the-art facilities that made the seminar
both engaging and effective.}

\subsection{Srozumitelnost}\label{srozumitelnost-1}

V předávání informací studující vidí závažné systémové nedostatky.
Uvítali by proaktivní poskytování informací o kurzech (forma, požadavky,
jazyk výuky a materiálů). Studující v angličtině mají pocit nerovného
přístupu k informacím, financování, projektům a plné formě výuky.
Informační servis často poskytují schopní jednotlivci (administativní i
akademičtí pracovníci), ale ne vždy jsou snadno zastupitelní.

\emph{From my point of view, there should be more information passed on
proactively to students. Especially in the second year, I had to ask for
everything I wanted to know about the courses I signed up for. Will
there be any lessons? Are they online or in person? What is required to
pass the course? I had to e-mail professors and ask all these questions
by myself. In all three courses I signed up for in the second year,
there were no lessons. I was just given an assignment that I had to
solve by myself without any prior teaching input.}

\emph{Students who do not speak Czech have to struggle a bit (difficulty
in accessing research funds, difficulty in being included in research
projects, different treatment by some professors: this last point in
particular was very relevant as the quality of teaching suffered greatly
due to the reluctance to teach courses in English {[}but this may have
been due to my not realising that the programme in English did not
include courses{]})}

\emph{I would improve access to courses, funding opportunities and
research projects for non-Czech speakers. Otherwise, it should be
clearly stated before enrolment that not all of the above is available
in English.}

\emph{Again, difficulties in communicating with professors influenced
the skills I wanted to develop at UWB.}

\emph{My supervisor is the most valuable person for my studies. I
receive any information I would like to know about my studies from her.
In the first year, I also received relevant administrative information
about the study program from the Dean's secretary. However, she resigned
and there was no one to substitute her one-by-one.}

\subsection{Finance}\label{finance-1}

Studující pociťují finanční těžkosti. Očekávají, že stipendium se bude
blížit běžnému příjmu a nad současným stavem pociťují zklamání. (pouze 2
odpovědi)

\emph{While I am grateful for the opportunities, I think improvements
could be made in the timely disbursement of scholarship funds to avoid
financial uncertainties.}

\emph{One thing I feel is not enough for the students, especially who
are married and has the responsibility of feeding his family and parents
- it is financial coverage or stipend \ldots.so you should improve it
according the Market \ldots\ldots{} }

\subsection{Sebevědomí}\label{sebevux11bdomuxed}

(pouze 1 otevřená odpověď)

\emph{I work at a German university that has a long partnership with ZCU
and we do many Czech-German projects beyond my doctoral studies
together. I developed the teaching competencies at my workplace at a
German university. I developed my project management and administration
skills, as well as the non-academic career development at my former and
my current workplace.}




\end{document}
